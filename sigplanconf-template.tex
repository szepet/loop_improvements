%-----------------------------------------------------------------------------
%
%               Template for sigplanconf LaTeX Class
%
% Name:         sigplanconf-template.tex
%
% Purpose:      A template for sigplanconf.cls, which is a LaTeX 2e class
%               file for SIGPLAN conference proceedings.
%
% Guide:        Refer to "Author's Guide to the ACM SIGPLAN Class,"
%               sigplanconf-guide.pdf
%
% Author:       Paul C. Anagnostopoulos
%               Windfall Software
%               978 371-2316
%               paul@windfall.com
%
% Created:      15 February 2005
%
%-----------------------------------------------------------------------------


\documentclass[preprint]{sigplanconf}

% The following \documentclass options may be useful:

% preprint       Remove this option only once the paper is in final form.
%  9pt           Set paper in  9-point type (instead of default 10-point)
% 11pt           Set paper in 11-point type (instead of default 10-point).
% numbers        Produce numeric citations with natbib (instead of default author/year).
% authorversion  Prepare an author version, with appropriate copyright-space text.

\usepackage{amsmath}
\usepackage[english]{babel}
\addto\captionsenglish{\renewcommand{\figurename}{Example}}
\usepackage{listings}  
\usepackage{color}
\usepackage[inline]{enumitem}

\definecolor{mygreen}{rgb}{0,0.6,0}
\definecolor{mygray}{rgb}{0.5,0.5,0.5}
\definecolor{mymauve}{rgb}{0.58,0,0.82}
\definecolor{mygrayback}{rgb}{0.9,0.9,0.9}
\lstset{ %
	backgroundcolor=\color{mygrayback},   % choose the background color; you must add \usepackage{color} or \usepackage{xcolor}; should come as last argument
	basicstyle=\small,        % the size of the fonts that are used for the code
	breakatwhitespace=false,
	breaklines=true, 
	captionpos=b,
	commentstyle=\color{mygreen},    % comment style
	extendedchars=true,              % lets you use 
	%frame=single,	                   % adds a frame 
	keepspaces=true,                 % keeps spaces in 
	keywordstyle=\color{blue},       % keyword style
	language=C++,                 %
	numbers=left,                    % where to put the line-numbers; possible values are (none, left, right)
	numbersep=5pt,                   % how far the line-numbers are from the code
	numberstyle=\footnotesize\color{mygray}, % the style that is used for the line-numbers
	rulecolor=\color{black},         % if not set, the frame-color may be changed on line-breaks within not-black text (e.g. comments (green here))
	showspaces=false,                % show spaces everywhere adding particular underscores; it overrides 'showstringspaces'
	showstringspaces=false,          % underline spaces within strings only
	showtabs=false,                  % show tabs within strings adding particular underscores
	stepnumber=1,                    % the step between two line-numbers. If it's 1, each line will be numbered
	stringstyle=\color{mymauve},     % string literal style
	tabsize=2,	                   % sets default tabsize to 2 spaces
	title=\lstname                   % show the filename of files included with \lstinputlisting; also try caption instead of title
}

\newcommand{\cL}{{\cal L}}
\hyphenation{Exploded-Graph}

\begin{document}
\lstset{language=C++}
\special{papersize=8.5in,11in}
\setlength{\pdfpageheight}{\paperheight}
\setlength{\pdfpagewidth}{\paperwidth}

\conferenceinfo{CONF'yy}{Month d--d, 20yy, City, ST, Country}
\copyrightyear{2018}
\copyrightdata{978-1-nnnn-nnnn-n/yy/mm}\reprintprice{\$15.00}
\copyrightdoi{nnnnnnn.nnnnnnn}

% For compatibility with auto-generated ACM eRights management
% instructions, the following alternate commands are also supported.
%\CopyrightYear{2016}
%\conferenceinfo{CONF'yy,}{Month d--d, 20yy, City, ST, Country}
%\isbn{978-1-nnnn-nnnn-n/yy/mm}\acmPrice{\$15.00}
%\doi{http://dx.doi.org/10.1145/nnnnnnn.nnnnnnn}

% Uncomment the publication rights used.
%\setcopyright{acmcopyright}
\setcopyright{acmlicensed}  % default

%\setcopyright{rightsretained}

%\titlebanner{banner above paper title}        % These are ignored unless
%\preprintfooter{short description of paper}   % 'preprint' option specified.

\title{Improved Loop Execution Modeling in the Clang Static Analyzer}
	%\titlenote{with optional title note}}

\authorinfo{Name1\thanks{with optional author note}}
           {Affiliation1}
           {Email1}
\authorinfo{Name2 \and Name3\thanks{with optional author note}}
           {Affiliation2/3}
           {Email2/3}

\maketitle

\begin{abstract}
The LLVM Clang Static Analyzer is a source code analysis tool to find bugs in C, C++, and Objective-C programs using symbolic execution, i.e. it simulates possible execution paths of the code. Currently, loop simulation is somewhat naive (but efficient), unrolling the loops for a predefined constant number of times. However, this approach can result in a loss of coverage in various cases.

This study aims to introduce two alternative approaches, which can extend the current method and can be applied simultaneously:
\begin{enumerate*} [label={(\arabic*)}, noitemsep]
	\item with applied heuristics we determine loops worth to fully unroll, and
	\item we use a widening mechanism to simulate an arbitrary iteration step.
\end{enumerate*}
These methods were evaluated on numerous open source projects, and were proved to increase coverage in most of the cases. This work also established an infrastructure for future loop modeling improvements.
\end{abstract}

% 2012 ACM Computing Classification System (CSS) concepts
% Generate at 'http://dl.acm.org/ccs/ccs.cfm'.

% \ccsdesc[500]{Software and its engineering~General programming languages}
% \ccsdesc[300]{Theory of computation~Program analysis}
% end generated code

% Legacy 1998 ACM Computing Classification System categories are also
% supported, but not recommended.
%\category{CR-number}{subcategory}{third-level}[fourth-level]
%\category{D.3.0}{Programming Languages}{General}
%\category{F.3.2}{Logics and Meanings of Programs}{Semantics of Programming Languages}[Program analysis]

\keywords
keyword1, keyword2 TODO

\section{Introduction}
The Clang Static Analyzer finds bugs by performing symbolic execution on the source code. During symbolic execution, the program is being interpreted, on a function-by-function basis, without any knowledge about the runtime environment. It builds up and traverses an inner model of the execution paths, called \texttt{ExplodedGraph}, for every analyzed function. A node of this graph (called ExplodedNode) contains a \texttt{ProgramPoint} (which determines the location) and a \texttt{ProgramState} (which contains all known information at that point). Its paths from the root to the leaves model the different execution paths of the analyzed function. Whenever the execution encounters a branch, a corresponding branch will be created in the \texttt{ExplodedGraph} during the simulated interpretation. Hence, branches lead to an exponential number of \texttt{ExplodedNode}s.

This combinatorial explosion is handled in the Static Analyzer by stopping the analysis when given conditions are fulfilled. Terminating the analysis process may cause loss of potential true positive results, but it is indispensable for maintaining a reasonable resource consumption regarding the memory and CPU usage. These conditions are modeled by the concept of a \textit{budget}.

The budget is a collection of limitations on the shape of the \texttt{ExplodedGraph}. These limitations include:
\begin{enumerate}
	\item The maximum number of traversed nodes in the \texttt{Exploded} \texttt{Graph}. If this number is reached, then the analysis of the simulated function stops.
	\item The size of the simulated call stack. When a function call is reached, the analysis continues in its body as if it was inlined to the place of call (interprocedural). There are several heuristics that may control the behavior of the inlining process. For example, excessively large functions are not inlined at all, and functions below a specified length limit are ignored while counting the call stack size.
	\item The number of times a function is inlined. The idea behind this constraint is that the more a function is analyzed, the less likely it is that a bug will appear in it. If this number is reached, then that function will not be inlined again in this \texttt{ExplodedGraph}.
	\item The number of times a basic block is processed during the analysis. This constraint limits the number of loop iterations. When this threshold is reached, the currently analyzed execution path will be aborted.

	The budget expression can be used in two ways. Sometimes it means the collection of the limitations above, sometimes it refers to one of these limitations. This will always be distinguishable from the context.
\end{enumerate}


\section{Motivation}
As already mentioned in the introduction, the analyzer handles loops quite simply in its current state. More precisely, it unrolls them 4 times by default and then cuts the analysis of the path where the loop would have been unrolled more than 4 times.

Loss in code coverage is one of the problems with this approach to loop modeling. Specifically, in cases where the loop is statically known to make more than 4 steps, the analyzer does not analyze the code following the loop. Thus, naive loop handling  could lead to entirely unchecked code. An example demonstrating the above described situation:

\lstset{language=C,caption={Since the loop condition is known at every iteration, the analyzer will not check the 'rest of the function' part in the current state.},label=UnrollMot}
\begin{lstlisting}
 void foo() {
   int arr[6];
   for (int i = 0; i < 6; i++){
     arr[i] = i;
   }
   /* rest of the function */
 }
\end{lstlisting}

According to the budget rule concerning the number of basic block visits, the analysis of the loop stops in the fourth iteration even if the loop condition is simple enough to see that unrolling the whole loop would be relatively straightforward. Running out of budget implies, in this case, that the rest of the function body remains unanalyzed, which may lead to not finding potential bugs.

Another problem can be seen on the following example:
\lstset{language=C++,caption={The loop condition is unknown but the analyzer will not generate a simulation path where n $\ge$ 4 (which can result coverage loss).},label=WidenMot}
\begin{lstlisting}
 int num();
 void foo()
 {
   int n = 0;
   for (int i = 0; i < num(); ++i) {
      ++n;
   }
   /* rest of the function, n < 4 */
 }
\end{lstlisting}

This code fragment results in an analysis that keeps track of the values of variables \texttt{n} and \texttt{i} (this information is stored in the \texttt{State}). In every iteration of the loop the values are updated accordingly. Note that updating the \texttt{State} means a new node insertion in the \texttt{ExplodedGraph} with new values. Since the body of the \texttt{num()} function is unknown, the analyzer cannot determine its return value. Thus, it is considered unknown, causing the graph to split to two branches. The first branch belongs to the symbolic execution of the loop body assuming that the loop condition is true. The other one simulates the case where the condition is false and the execution continues after the loop. This process is done for every loop iteration, however, the 4th time assuming the condition is true, the path will be cut short according to the budget rule.

Although the analyzer generates paths to simulate the code after the loop in the above described case, yet the value of variable \texttt{n} will be always less than 4 on these paths and the rest of the function will only be checked assuming this constraint. This can result in coverage loss as well, since the analyzer will ignore the paths where n is more than 4.

\section{Proposed Solution}
%The current loop handling method in the Clang Static Analyzer is too strict.
In this section we present two solutions to resolve the previously mentioned limitations on symbolic execution of loops in the Clang Static Analyzer. It is important to note that these enhancements are incremental in the sense that on examples which are too complex to handle at the moment, we fall back to the original method. For the sake of simplicity in the following examples a "division by zero" will illustrate the bug we intend to find.

\subsection{Loop Unrolling Heuristics}

Loop unrolling means we have identified heuristics and patterns (such as loops with small number of branches and small known static bound) in order to find specific loops which are worth to be completely unrolled. This idea is inspired by the following example:

\lstset{language=C++,caption={Complete unrolling of the loop makes possible to find the division by zero error.},label=UnrollEx}
\begin{lstlisting}
 void foo() {
   for (int i = 0; i < 6; i++){
     /* simple loop which does not
        change 'i' or split the state */
   }
   int k = 0;
   int l = 2/k; /* division by zero */
 }
\end{lstlisting}

In the current solution a loop has to fulfill the following conditions in order to be unrolled:
\begin{enumerate}  
	\item The loop condition should arithmetically compare a variable -- which is known at the beginning of the loop -- to a literal (like: \texttt{i $<$ 6} or \texttt{6 $\ge$ i}).
	\item The loop should only change the loop variable in its body once and the difference needs to be constant. (This way we can estimate the maximum number of steps.)
	\item The estimated number of steps should be less than 128. (We still do not want to simulate loops which take thousands of steps because they could single-handedly exhaust the budget.)
	\item The loop must not generate new branches or use \texttt{goto} statements.
\end{enumerate}

Using this method we can successfully find the bug in the above example.

\subsection{Loop Widening}

The final aim of widening is quite the same as of unrolling, to increase the coverage of the analysis. However, it accomplishes it in a very different way. During widening the analyzer simulates the execution of an arbitrary number of iterations. There is already a solution which achieves this behavior by discarding all known information before the last step of the loop. So, the analyzer creates paths for the first 3 steps and simulates them as usual, but the widening (i.e. invalidating) happens before the 4th step in order to preserve the first 3 precise simulation branches.

This way the coverage will increase but this method can easily result in too much false positives. Consider the following example:

\lstset{language=C++,caption={Invalidating every known information (even the ones which are not modified by the loop) can easily result false positives.},label=WidenProb}
\begin{lstlisting}
 int num();
 void foo() {
   bool b = true;
   for (int i = 0; i < num(); ++i) {
     /* does not change 'b' */
   }
   int n = 0;
   if (b)
     n++;
   n = 1/n; /* false positive:
               division by zero */
   }
 }
\end{lstlisting}
In this case the analyzer will create and check the impossible path where the variable \texttt{b} is false, so \texttt{n} is not incremented which leads to a division by zero error. Since this execution path would never be performed while running the analyzed program, it is considered a false positive. My aim was to give a more precise approach for widening.

The main principle is that we try to continue the analysis after the block visiting budget is exhausted and invalidate the information only on the variables which are possibly modified by the loop.

For this I developed a solution which checks every possible way in which a variable can be modified in the loop. Then it evaluates these cases and if it encounters a modified variable which cannot be handled by the invalidation process (e.g. a pointer variable), hence the loop will not be widened and we return to the conservative method. This mechanism ensures that we do not create nodes containing invalid states.

This approach helps us cover cases and find bugs like the example below shows, and still not report false positives represented by the previous example.

\lstset{language=C++,caption={Invalidating the information on only the possible changed variables can result higher coverage (while limiting the number of the found false positives).}, label=WidenEx}
\begin{lstlisting}
 int num();
 void foo() {
   int n = 0;
   for (int i = 0; i < num(); ++i) {
     ++n;
   }
   if (n > 4) {
     int k = 0;
     k = 1/k;  /* division by zero error */
   }
}
\end{lstlisting}

We find the bug since we invalidate all known information on variable \texttt{n} (and \texttt{i} as well). This causes the analyzer to create a branch where it checks the body of the \texttt{if} statement and finds the bug.

However, this solution has its own limitations when dealing with nested loops. Consider the following case:

\begin{lstlisting}
 int num();
 void foo() {
   int n = 0;
   for (int i = 0; i < num(); ++i) {
    ++n;
    for (int j = 0; j < 4; ++j) {
      /* body that does not change 'n' */
    }
   }
   /* rest of the function, n <= 1 */
 }
\end{lstlisting}

In this scenario, when the analyzer first steps into the outer loop (i.e. assumes that \texttt{i < num()} is true) and encounters the inner loop, then it consumes its own block visiting budget. (This implies that it will be widened, although in this case only the inner loop counter (\texttt{j}) information is discarded.) Afterwards we move on to the next iteration, and assume that we are on the path where the outer loop condition is true again. Because we already exhausted the budget in the previous iteration, at the next visit of the first basic block of the inner loop (the condition) this path will be completely cut off and not analyzed. This results in the outer loop not reaching the number of steps where it would been widened. Furthermore, the outer loop will not even reach the 3rd step and the 2nd is stopped at in its body as well (as described above). This causes the problem that even though we use the loop widening method, we will analyze the rest of the function with the assumption \texttt{n <= 1}.

In order to deal with the above described nested loop problem, I have implemented a replay mechanism. Whenever we encounter an inner loop which already consumed its budget, we replay the analysis process of the current step of the outer loop but perform a widening first. This ensures the creation of a path that assumes the condition to be false and simulates the execution after the loop with any possibly changed information discarded. This way the analyzer will not exclude some prunable paths because of the simple loop handling, which solves the problem.

As another note to the widening process, it makes sense to analyze the branch where the condition is true with the widened \texttt{State} as well. The following example shows a case where this is useful:

\begin{lstlisting}
int num();
void foo() {
 int n = 0;
 int i;
 for (i = 0; i < num(); ++i) {
   if (i == 7) {
     break;
   }
   for (int j = 0; j < 4; ++j) {/* */}
 }
 int n = 1 / (7 - k); /* possible division by zero */
}
\end{lstlisting}

This way the analyzer will produce a path where the value of \texttt{i} is known to be 7, so it will be able find the possible division by zero error.

\section{Evaluation}

The effect of the described loop modeling approaches was measured on various C/C++ open source projects.

\subsection{Loop Unrolling}
\subsection{Loop Widening}
\section{Conclusion}
\section{Future work}
\acks

Acknowledgments, if needed.

% The 'abbrvnat' bibliography style is recommended.

\bibliographystyle{abbrvnat}

% The bibliography should be embedded for final submission.

\begin{thebibliography}{}
\softraggedright

\bibitem[Smith et~al.(2009)Smith, Jones]{smith02}
P. Q. Smith, and X. Y. Jones. ...reference text...

\end{thebibliography}


\end{document}
